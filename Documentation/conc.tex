\section{Conclusion}

Creating a Convolutional Neural Network through this lab exercise have provided valuable insights into what it takes to design a machine learning for deep learning and image processing. By repetitive and meticulous experimentation, we were able to figure out what was most effective in terms of architectural configurations, activation functions, optimization techniques, etc. to achieve optimal performance for detecting hair types.

We have witnessed firsthand how small adjustments can lead to significant improvements or detriments in performance, truly encapsulating the ideal of balance between complexity and generalization to avoid overfitting and possess the ability for generalization. 

Additionally, this lab exercise provided insights for CNN itself especially how convolution layers, pooling operations, and the hierarchical feature extraction process works. This gives us the edge on the skills for designing image processing models and how to really improve them. 

In the future, the knowledge and skills acquired from this lab exercise is a solid foundation for further exploration in deep learning. As seen in one experimentation, models from previous studies (i.e. the U-Net architecture) are not only understood but also implemented. As we continue this journey, we will learn more and more about the field of machine learning and use it to deal with real-world challenges with ingenuity, creativity, and efficiency. 
